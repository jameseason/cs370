\documentclass[11pt]{article}
\usepackage{setspace}
\usepackage[letterpaper,top=1in,bottom=1in,left=1.25in,right=1.25in]{geometry}
\usepackage[parfill]{parskip}

\title{Evolutionary System Development and the Classic Development Phases}
\author{James Eason}
\date{22 January 2018}

\begin{document}
\maketitle

\thispagestyle{empty}

\begin{doublespace}
  \section{Introduction}

Despite the opinions of early computer scientists such as Fritz Bauer that 
``a rigorous engineering approach is needed'' to deliver dependable and usable software, 
this lengthy process of thorough preplanning with detailed specifications
is not a practical software development method \cite{denning2008}. Denning et al.\ argue that 
using agile, an evolutionary design approach, is superior to preplanning regardless
of the size or desired reliability of the system \cite{denning2008}. The classic development phases, 
while still relevant, are less rigid in evolutionary system development compared to traditional preplanning.

  \section{Background}

The traditional preplanning approach is flawed because it is not adaptable. 
The FBI virtual case file and the replacement FAA air traffic control system are 
two examples of systems which failed because the development time was longer
than the environment change time \cite{denning2008}. Due to the lack of adaptability during the development
process, these systems were obsolete before they could be released \cite{denning2008}.

Evolutionary system development, alternatively, does not include a formal preplanning process,
only a general notion of the system architecture \cite{denning2008}. It allows adaptability for changing requirements and uses a modular approach, where components that do not work well are abandoned \cite{denning2008}. This allows and encourages 
risks which have the potential to yield superior results compared to a more conservative preplanned approach.

  \section{Analysis}
  
In classic development, the development phases are requirements, specifications, design, implementation, integration, and maintenance \cite{beck118}, and they are followed in order. 
In agile development, the phases must be more flexible and adaptable. Maintenence, for example, is 
done throughout the development process, so it is less of a distinct phase. Additionally, since requirements may be changed in agile development,
 the rest of the phases must be repeated with every change. As a result, there may not be a single phase which defines the 
progress or status of the project.  


  \section{Conclusion}

Overall, evolutionary system development is superior to traditional preplanning due to its adaptability. Both incorporate the classic development phases, but evolutionary system development does so in a less rigid way.

\end{doublespace}

\bibliographystyle{ieeetr}
\bibliography{sebibliography}

\end{document}
